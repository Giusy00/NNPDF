
\section{Filter}

The filter project reads experimental data sets from articles format,
performs cross-checks on errors and creates the following files for
each experiment: 
\begin{itemize}
\item \texttt{\textbf{OBS\_<setname>.dat}} $\Rightarrow$ Contains the
experimental points 
\item \texttt{\textbf{COVMAT\_<setname>.dat}}\textbf{ }$\Rightarrow$ Contains
the covariance matrix of experimental data 
\item \texttt{\textbf{INVCOVMAT\_<setname>.dat}}\textbf{ }$\Rightarrow$
Contains the inverse of the covariance matrix 
\end{itemize}
This operation must be done once for all experiments and repeated
only in case new data is introduced or modified. Then, the generated
files should be located in \texttt{data/setname} folders respectively.

\subsection{Running filter}

Filter takes as input the configuration file location. By default
it uses config/config.ini as configuration file. For example you can
run filter as

\begin{lstlisting}
./filter example.ini
\end{lstlisting}


No interaction with the filter code is needed, you just modify the
variables in the configuration file presented in Table (\ref{tab:Configuration-information.}).

\begin{table}[H]
\begin{centering}
\begin{tabular}{|c|c|c|}
\hline 
\multicolumn{1}{|c}{\texttt{\textbf{{[}data{]}}}} & \multicolumn{1}{c}{Type} & Description\tabularnewline
\hline 
\hline 
\texttt{\textbf{RESULTSDIR}} & \texttt{string} & %
\begin{minipage}[t]{0.5\columnwidth}%
The directory where output will be located.%
\end{minipage}\tabularnewline
\hline 
\hline 
\texttt{\textbf{{[}datasets{]}}} &  & \tabularnewline
\hline 
\hline 
\texttt{\textbf{DATASET}} & \texttt{string} & %
\begin{minipage}[t]{0.5\columnwidth}%
you just write line per line the name of data sets which you want
to include in the validphys computation, from Table (\ref{tab:FK-tables-status.}).%
\end{minipage}\tabularnewline
\hline
\hline
\texttt{\textbf{{[}filter{]}}} & & \tabularnewline
\hline
\hline
\texttt{\textbf{USET0}} & \texttt{int(0,1)} & %
\begin{minipage}[t]{0.5\columnwidth}%
if 0 filter will produce the experimental covariance matrix, otherwise the t0 covariance matrix will be generated.%
\end{minipage}\tabularnewline
\hline
\texttt{\textbf{T0PDFSET}} & \texttt{string} & %
\begin{minipage}[t]{0.5\columnwidth}%
PDF set to generate the t0 covariance matrix (no .LHgrid needed)%
\end{minipage}\tabularnewline
\hline
\hline
\texttt{\textbf{{[}evolution{]}}} & & \tabularnewline
\hline
\hline
\texttt{\textbf{PTORD}} & \texttt{int(1,2)} & %
\begin{minipage}[t]{0.5\columnwidth}%
the perturbation order%
\end{minipage}\tabularnewline
\hline
\texttt{\textbf{VFNS}} & \texttt{string} & %
\begin{minipage}[t]{0.5\columnwidth}%
Flavor scheme: FFN0, FFNS, ZMVN, GMVN%
\end{minipage}\tabularnewline
\hline
\texttt{\textbf{VFNSTYPE}} & \texttt{string} & %
\begin{minipage}[t]{0.5\columnwidth}%
FONLL (A, B or C), only for GMVN%
\end{minipage}\tabularnewline
\hline
\hline
\texttt{\textbf{{[}kincuts{]}}} & & \tabularnewline
\hline
\hline
\texttt{\textbf{IQ2CUT}} & \texttt{int} & %
\begin{minipage}[t]{0.5\columnwidth}%
depends on the fk table cuts.%
\end{minipage}\tabularnewline
\hline
\texttt{\textbf{NPARSAT}} & \texttt{int} & %
\begin{minipage}[t]{0.5\columnwidth}%
depends on the fk table cuts.%
\end{minipage}\tabularnewline
\hline
\texttt{\textbf{PARSAT}} & \texttt{array double} & %
\begin{minipage}[t]{0.5\columnwidth}%
depends on the fk table cuts.%
\end{minipage}\tabularnewline
\hline
\texttt{\textbf{IREG}} & \texttt{int} & %
\begin{minipage}[t]{0.5\columnwidth}%
depends on the fk table cuts.%
\end{minipage}\tabularnewline
\hline
\texttt{\textbf{Q2MINCUT}} & \texttt{double} & %
\begin{minipage}[t]{0.5\columnwidth}%
depends on the fk table cuts.%
\end{minipage}\tabularnewline
\hline
\texttt{\textbf{Q2MIN}} & \texttt{double} & %
\begin{minipage}[t]{0.5\columnwidth}%
depends on the fk table cuts.%
\end{minipage}\tabularnewline
\hline
\texttt{\textbf{W2MIN}} & \texttt{double} & %
\begin{minipage}[t]{0.5\columnwidth}%
depends on the fk table cuts.%
\end{minipage}\tabularnewline
\hline
\end{tabular}
\par\end{centering}

\caption{\label{tab:Configuration-information.}Configuration information.}
\end{table}

\subsection{Interpreting results}

After running filter, you will find a: 
\begin{itemize}
\item \texttt{\textbf{filter.log}} file in \texttt{RESULTSDIR}, which contains
the configuration used to generated filter. 
\item \texttt{\textbf{RESULTSDIR/<config name>/filter}} folder containing all the generated
{*}.dat files, for each experiment set. \end{itemize}

