
\section{Buildmaster}

The buildmaster project reads experimental data sets from articles format,
performs cross-checks on errors and creates the following files for
each experiment: 
\begin{itemize}
\item \texttt{\textbf{DATA\_<setname>.dat}} $\Rightarrow$ Contains the
experimental points + uncertainty information
\end{itemize}
This operation must be done once for all experiments and repeated
only in case new data is introduced or modified. Then, the generated
files should be located in \texttt{data/setname} folders respectively.

\subsection{Adding a new experiment}
\begin{itemize}
  \item go to data/ and create a folder with the experiment (dataset) name, you should put the data files inside a rawdata folder (look at data/CMSWEASY840PB)
  \item open include/commondata.h and add a void filter<experiment name>() method
  \item open src/commondata.cc, update the void CommonData::CreateCommonFormat()
  \item open src/commondata.cc and implement the filter<experiment name> method.
  \item to activate/deactivate the experiment you should use the configuration file.
\end{itemize}

\subsection{Running buildmaster}

Buildmaster takes as input the configuration file location. For example you can
run buildmaster as

\begin{lstlisting}
./buildmaster example.ini
\end{lstlisting}
No interaction with the filter code is needed.

\subsection{Interpreting results}

After running filter, you will find a: 
\begin{itemize}
\item \texttt{\textbf{buildmaster.log}} file in \texttt{RESULTSDIR}, which contains
the configuration used to generated filter. 
\item \texttt{\textbf{RESULTSDIR/<config name>/buildmaster}} folder containing all the generated
{*}.dat files, for each experiment set. \end{itemize}
Then you should copy the generated DATA\_<experiment>.dat file into data/<experiment> folder and upload to svn. Users should never run buildmaster.

